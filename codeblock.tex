\documentclass[a4j]{jarticle}
\usepackage[dvipdfmx]{graphicx}
\usepackage[dvipdfmx]{color}
\usepackage{listings,jlisting}
\usepackage{multicol}
% Different color code blocks can be used by changing the RGB values
\definecolor{backcolor}{rgb}{0.95,0.95,0.92}
\definecolor{commentcolor}{rgb}{0,0.6,0}
\definecolor{keycolor}{rgb}{0,0,1}
\definecolor{purplecolor}{rgb}{0.58,0,0.82}
\definecolor{graycolor}{rgb}{0.6,0.6,0.6}
\lstset{ 
language={C}, 
backgroundcolor=\color{backcolor},
basicstyle={\small},
identifierstyle={\small},
commentstyle={\small\ttfamily \color{commentcolor}},
keywordstyle=\color{keycolor},
stringstyle=\color{purplecolor},
numberstyle=\color{graycolor},
ndkeywordstyle={\small}, 
frame={tb}, 
breakatwhitespace=false,         
breaklines=true,                 
keepspaces=true,                 
numbers=left,       
numbersep=5pt,                  
showspaces=false,                
showstringspaces=false,
showtabs=false,                  
tabsize=2,
morecomment=[l]{//} 
} 

\begin{document}

\section{texファイルに直接書き込む}
\begin{lstlisting}[caption=helloworld.c, label=hello]
#include<stdio.h>
int main(){
   /* Hello world! を出力する */
   printf("Hello world!");
}
\end{lstlisting}
ソースコード\ref{hello}はHello worldを出力するプログラムである.

\section{コードのファイルパスを指定して利用する}
\lstinputlisting[caption=hoge.c, label=hoge]{hoge.c}
ソースコード\ref{hoge}は文字列this is hogeとhogeの値を出力するプログラムである.

\section{行数を非表示にする、異なる言語を表示する}
\lstinputlisting[language=sh, numbers=none]{sample.sh}
コマンドの実行結果は上の通りである.

\end{document}

